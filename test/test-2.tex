\usemodule[zhfonts]

% you can set 'sans' and 'mono' respectively 
% and their 'regular', 'bold', 'italic', 'bolditalic'
\setupzhfonts[serif][regular=adobesongstd]
\setupzhfonts[latin,serif][regular=texgyrepagella]

\zhfonts[rm,11pt]

\starttext

\title{出师表}

“有泪都成血!”。

先帝创业未半而中道崩殂,今天下三分,益州疲弊,此诚危急存亡之秋也。然侍卫之臣不懈于内,忠志之士忘身于外者,盖追先帝之殊遇,欲报之于陛下也。诚宜开张圣听,以光先帝遗德,恢弘志士之气,不宜妄自菲薄,引喻失义,以塞忠谏之路也。

宫中府中,俱为一体,陟罚臧否,不宜异同。若有作奸犯科及为忠善者,宜付有司论其刑赏,以昭陛下平明之理,不宜偏私,使内外异法也。

侍中侍郎郭攸之、费祎、董允等,此皆良实,志虑忠纯,是以先帝简拔以遗陛下。愚以为宫中之事,事无大小,悉以咨之,然后施行,必能裨补阙漏,有所广益。

将军向宠,性行淑均,晓畅军事,试用于昔日,先帝称之曰能,是以众议举宠为督。愚以为营中之事,悉以咨之,必能使行阵和睦,优劣得所。

亲贤臣,远小人,此先汉所以兴隆也;亲小人,远贤臣,此后汉所以倾颓也。先帝在时,每与臣论此事,未尝不叹息痛恨于桓、灵也。侍中、尚书、长史、参军,此悉贞良死节之臣,愿陛下亲之信之,则汉室之隆,可计日而待也。

臣本布衣,躬耕于南阳,苟全性命于乱世,不求闻达于诸侯。先帝不以臣卑鄙,猥自枉屈,三顾臣于草庐之中,咨臣以当世之事,由是感激,遂许先帝以驱驰。后值倾覆,受任于败军之际,奉命于危难之间,尔来二十有一年矣。

先帝知臣谨慎,故临崩寄臣以大事也。受命以来,夙夜忧叹,恐托付不效,以伤先帝之明,故五月渡泸,深入不毛。今南方已定,兵甲已足,当奖率三军,北定中原,庶竭驽钝,攘除奸凶,兴复汉室,还于旧都。此臣所以报先帝而忠陛下之职分也。至于斟酌损益,进尽忠言,则攸之、棉、允之任也。

愿陛下托臣以讨贼兴复之效,不效则治臣之罪,以告先帝之灵。若无兴德之言,则责攸之、祎、允等之慢,以彰其咎;陛下亦宜自谋,以咨诹善道,察纳雅言,深追先帝遗诏。臣不胜受恩感激。

今当远离,临表涕零,不知所言。

\stoptext
